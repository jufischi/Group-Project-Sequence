\documentclass{article}


\usepackage{arxiv}

\usepackage[utf8]{inputenc} % allow utf-8 input
\usepackage[T1]{fontenc}    % use 8-bit T1 fonts
\usepackage{hyperref}       % hyperlinks
\usepackage{url}            % simple URL typesetting
\usepackage{booktabs}       % professional-quality tables
\usepackage{amsfonts}       % blackboard math symbols
\usepackage{nicefrac}       % compact symbols for 1/2, etc.
\usepackage{microtype}      % microtypography
\usepackage{lipsum}
\usepackage{graphicx}
\usepackage{todonotes}

\title{The Sankoff Algorithm for Phylogeographics}


\author{
  Julia Fischer \\
  Universität Tübingen \\
  \texttt{juli.fischer@student.uni-tuebingen.de} \\
  6039174 \\
  \And
  Peter Heringer \\
  Universität Tübingen \\
  \texttt{peter.heringer@student.uni-tuebingen.de} \\
  6109174 \\
  \And 
  Michael Mederer \\
  Universität Tübingen \\
  \texttt{michael.mederer@student.uni-tuebingen.de} \\
  XXXXXXX \\
  \And 
  Felix Seidel \\
  Universität Tübingen \\
  \texttt{felix.seidel@student.uni-tuebingen.de} \\
  5969276 \\
}

\begin{document}
\maketitle

\begin{abstract}
\end{abstract}


\section{Introduction}
Given a rooted tree $T$ with labeled leaves, the small parsimony problem is about
finding labels for the internal nodes of the tree such that the changes from an
internal nodes to all of its children are minimal
\cite{jonesIntroductionBioinformaticsAlgorithms2004}.
Among others, the Sankoff algorithm can be used to solve the small parsimony
problem \cite{sankoffMinimalMutationTrees1975}. 

Sankoff's algorithm is designed to use an already existing phylogenetic tree with labeled leaves and
some form of cost matrix to label the internal nodes in a way that minimizes the cost based on the
matrix and underlying leaves. For this the algorithm uses two phases a forward and a backward pass.
The forward pass fills for every node a list detailing what choosing a certain label would incur in
cost. It starts with the internal nodes just above the leaves and ends with the root. This is the
more calculation intensive part of the algorithm. The second part is the backward pass. It starts at
the root and ends with the internal nodes above the leaves. Its task is to choose the correct label
based on the cost lists. This results in the labeled tree.

The Sankoff algorithm is usually applied to the broader scope of
inferring phylogenetic trees using cost matrices that model the transition
between different DNA or RNA nucleotides
\cite{jonesIntroductionBioinformaticsAlgorithms2004}, i.e. solving the large parsimony problem. By
using different cost
matrices, the algorithm can also be used to model other biological questions,
for example in the Camin-Sokal-Parsimony \cite{caminMethodDeducingBranching1965}
or in the Dollo-Parsimony \cite{farrisPhylogeneticAnalysisDollo2022}. More
recently, the Sankoff algorithm was also used to infer the geographic origins of
the 2009 H1N1 pandemic using a distance matrix that represents the distances of
various international airports
\cite{reimeringPhylogeographicReconstructionUsing2020}. In this, the authors use
a phylogenetic tree that represents the relationship between different strains
of the virus and map each taxon to the airport that is closest to the sampling
location. Then, three different cost matrices are utilized:

\begin{enumerate}
  \item Equal distance: Airports have a distance of 1 to each other
  \item Geographic distance: The actual geographic distances between different
  airports is used.
  \item Effective distances: A infection is more likely to spread from
  airport $A$ to airport $B$ if there are many people traveling from $A$ to
  $B$. Thus, the effective distances are calculated by using passenger data
  for each combination of airports.
\end{enumerate}

Using those cost matrices for airports and for the corresponding countries, the
labels for the internal nodes of the tree are calculated and thus their
geographic position.

One of the advantages in using the Sankoff algorithm for this problem as opposed to more
conventional approaches is, as the authors describe a gain in performance. This is due to the fact
that the Sankoff algorithm is comparatively simple. Additionally, large parts can be implemented as
operations on matrices which can be done very quick in modern computers because of easy
parallelization and efficient libraries, such as BLAS \cite{lawsonBasicLinearAlgebra1979}. As
datasets can be quite large due to a large
amount of locations (in this case airports) and sequences having a fast
algorithm to even approximate the correct solution can be instrumental in getting a solution.

In the following report, we implement the method outlined in
\cite{reimeringPhylogeographicReconstructionUsing2020} and recreate the findings
by using the original data.

\section{Material and Methods}
The Sankoff algorithm was implemented using the Python
\cite{pythonsoftwarefoundationWelcomePythonOrg2023} and Numpy \cite{harrisArrayProgrammingNumPy2020}. Numpy was used to allow the use of
efficient matrix operations.

All visualizations (with the exception of visualizations from the original paper)\todo{Do we need
this?} were done using Python and Matplotlib \cite{MatplotlibVisualizationPython}. To accurately plot the maps and the locations of the
airports Geopandas \cite{GeoPandas12GeoPandas} was used. Additionally, we used the package airportsdata \cite{borsettiAirportsdataExtensiveDatabase2022} to get the locations of
airpots and pycountry \cite{theunePycountryISOCountry} to translate letter codes of countries.

\section{Results}



\subsection{Fr\'{e}chet distances}
\begin{itemize}
    \item Fr\'{e}chet tree distances used to quantify geographic differences between spread paths
    \item We compare our reconstructed trees with the ones from the paper
    \item by claculating discrete Fr\'{e}chet tree distances using geographic distances between locations
    \item method compares the paths of locations from the root of each leaf node, calculates discrete Fr\'{e}chet distances between them and corrects the distance for each node by the number of paths
    \item we use Rscript from the authors for calculation \cite{reimeringFrechetTreeDistance2018}
\end{itemize}

Our inferred trees versus the trees from the paper:
\begin{itemize}
    \item geographic distances: 123.8497
    \item effective distances: 0
\end{itemize}

The Fr\'{e}chet distance between our trees:
\begin{itemize}
    \item effective airport vs geographic airport: 65149
    \item effective country vs geographic country: 12307.03
    \item airport to country comparison not possible due to distance matrix
\end{itemize}


\section{Discussion}


\bibliographystyle{unsrt}  
\bibliography{references}  %%% Remove comment to use the external .bib file (using bibtex).
%%% and comment out the ``thebibliography'' section.

\end{document}

