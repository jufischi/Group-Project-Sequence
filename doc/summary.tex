\documentclass{article}
\usepackage[style=numeric]{biblatex}
\addbibresource{references.bib}

\title{Phylogeography using Sankoff}
\author{Julia Fischer 
\and 
Peter Heringer
\and
Michael Mederer
\and
Felix Seidel}

\begin{document}

\maketitle

In \cite{reimering2020phylogeographic}, the authors examine the geogprahic
origins of the 2009 H1N1 influenca pandemic by applying the Sankoff parsimony
method to a phylogenetic tree using a cost matrix built of the geographic
distances between the sampling locations for each leaf of the tree.

Each sample is mapped to the closest airport. From this, three different kinds
of distances are calculated:
\begin{enumerate}
    \item Equal distance: Each airport has a distance of 1 to the others. This
    makes the Sankoff-Algorithm behave like Fitch-Algorithm.
    \item Geographic distance: The actual geographic distances between different
    airports is used.
    \item Effective distances: A infection is more likely to spread from
    airport $A$ to airport $B$ if there are many people traveling from $A$ to
    $B$. Thus, the effective distances are calculated by using passenger data
    for each combination of airports.
\end{enumerate}

A phylogenetic tree is constructed from the sampled data, the method of choice
seems to be not relevant. Using that tree, the Sankoff parsimony algorithm is
used to infer the geographic positions of the internal nodes.

\printbibliography

\end{document}